\documentclass[a4paper,11pt,aps,secnumarabic,balancelastpage,amsmath,amssymb,nofootinbib,floatfix, table,twoside,openright]{report}
\usepackage{packages}
\usepackage[backend=biber,style=numeric,sorting=none]{biblatex}
\addbibresource{references.bib}


\begin{document}

\pagenumbering{roman}
\subfile{title.tex}

\hypersetup{linkcolor=black}
\tableofcontents
\clearpage

\subfile{0_introduction.tex}
\clearpage

\pagestyle{fancy}
\fancyhead[LE,RO]{\thepage}
\fancyhead[RE,LO]{\textit{\rightmark}}
\cfoot{}
\pagenumbering{arabic}

\subfile{1_single.tex}
\clearpage

\subfile{2_multi.tex}
\clearpage

\subfile{3_earthquakes.tex}
\clearpage

\printbibliography
\addcontentsline{toc}{chapter}{Bibliography}


\end{document}

\begin{comment}

CAPITOLO 1: Spiego il Burridge-Knopoff e l'analogo elettronico seguendo il paper (inductorless ecc).
    Faccio vedere l'implementazione sulla breadboard e confronto con la board (single block analysis).

CAPITOLO 2: Spiego l'analisi del caos seguendo il paper (chasing chaos).
    Faccio vedere il caos sulla board accoppiando vari blocchi.

CAPITOLO 3: Parlo dei terremoti, della legge di Gutenberg-Richter seguendo il paper (mia tesi hehehe).
    Magari metto il cellular automaton o altri risultati simili.
    Poi accoppio tutti i 25 con varie topologie e vedo cosa succede.

\end{comment}