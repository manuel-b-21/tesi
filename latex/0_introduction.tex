\chapter*{Abstract}
\addcontentsline{toc}{chapter}{Abstract}

The analysis of signals stemming from a physical system is crucial for the experimental investigation
of the underlying dynamics that drives the system itself. The field of time series analysis comprises
a wide variety of techniques developed with the purpose of characterizing signals and, ultimately, of
providing insights on the phenomena that govern the temporal evolution of the generating system.
An example of these techniques is given by spectral analysis: the use of Fourier or Laplace transforms
to bring time-domain signals into the more advantageous frequency space allows to disclose the key
features of linear systems. These procedures, however, are not helpful when dealing with nonlinearity.
Nonlinear systems usually exhibit interesting behaviors, such as self-sustained periodic oscillations
or quasi-periodic temporal evolutions. One of the most compelling consequences of nonlinearity is chaos.

Chaos is a dynamical regime characterized by unpredictability and lack of periodicity, despite being
generated by deterministic laws. Signals generated by chaotic dynamical systems appear as irregular:
the corresponding spectra are generally broad and flat, predictions of future values are demanding,
and the time evolution converges to a strange attractor with noninteger dimensionality.
For these reasons, linear techniques such as Fourier analysis often mistakenly classify chaotic signals
as noise.

Nonlinear time series analysis techniques act directly within the state space of the system under
investigation. However, experimentally, full access to a system's state space is not always available.
Often, only a scalar signal stemming from the dynamical system can be recorded, thus providing, upon
sampling, a scalar sequence. An important theorem by Takens states that it is possible to reconstruct
a surrogate of the original state space evolution starting from this time series. This is possible
due to the so-called time delay embedding: $m$-dimensional vectors are built by picking successive
elements of the scalar sequence delayed by a lag $L$. If the embedding parameters $(m,L)$ are
suitably chosen, the space constituted by the $m$-dimensional vectors is topologically identical
to the actual state space. Unfortunately, Takens' theorem does not provide any hint on how to find
the optimal embedding parameters. One way to solve this issue is to carry out the embedding procedure
for different values of $(m,L)$, searching for a set of values that correctly reconstructs the
dynamics.

The identification of chaos and its characterization require the assessment of dynamical invariants
that quantify the complex features of a chaotic system's evolution. There are two main invariant
quantities that are used to establish whether a system is chaotic or noisy.
The first one is the maximum Lyapunov exponent, which is a marker of unpredictability. The main
condition that determines that a system is chaotic is the sensitivity to initial conditions:
provided that two initial conditions are arbitrarily close in the state space, the distance
between the two diverges exponentially in time. The measure that quantifies this divergence is the
maximum Lyapunov exponent. The second invariant quantity is the correlation dimension, which is an
estimate of the non-integer dimension of the attractor; this measure highlights the unconventional
geometry of a chaotic system's state space.

An example of chaotic dynamics is given by earthquakes. Seismic faults are governed by Newton's law,
being thus a deterministic system. However, nonlinearity makes the temporal evolution of earthquakes
very hard to predict. In order to study the dynamics of such systems, it is necessary to involve
simpler models that reproduce the same properties as real earthquakes. A model that recreates these
features fairly well is the Burridge-Knopoff (BK) spring-block model. This consists in a two-dimensional
system of massive blocks interconnected by springs; each block is also connected to a moving surface
through another set of springs. Nonlinearity comes from the fact that the blocks lie on a rough
horizontal surface, which means that their velocity is weakened by some friction. Integrating the
differential equations that characterize this model allows us to compare the properties of this model
to the properties of real faults. The most important result achieved by this model is the
compliance with Gutenberg-Richter's law, which links the number of avalanches with the energy
released by the avalanche itself.

Despite the practicality of this model, a physical implementation of the Burridge-Knopoff model raises
several issues: the difficulty of measuring position and velocity for each block, the non-ideality
of springs, the problem of dealing with avalanches, just to name a few. In essence, an implementation
of the spring-block model is impossible outside of a computer.

One way to get around this question is to build some physical system which differential equations
are the same as the ones of the spring-block model. This can be done through an electronic circuit. By only
making use of resistances, capacitors, diodes and operational amplifiers, it is possible to design
a circuit that behaves in the same manner as a single BK block. The state variables
of the system are two voltages, $W$ and $V$, which are the analogue of position and velocity, respectively.
In absence of coupling between blocks, each block behaves as a simple oscillator. When coupling two or
more blocks instead, a seemingly chaotic behavior can be observed, if the parameters are suitably
chosen.

This circuit was first implemented and analyzed on a breadboard. However, the use of large electronic
components made the study of many coupled blocks impossible. An integrated board with 25 BK blocks
was then produced. By making use of the embedding procedure and other techniques, the chaotic dynamics
of this system can be resolved starting from the time series $W$. Analyzing the chaotic dynamics of
several coupled blocks will be the final aim of this work.

The present thesis is organized as follows. The mechanical Burridge-Knopoff model and its electronic
implementation are presented in Chapter~\ref{chap: electronic analog of bk}. A characterization of the
single and double block behavior is also provided, hinting at the possibility of chaotic dynamics.
In Chapter~\ref{chap: chaos} a general overview of chaos is given. Both the mathemathical definition
of chaos and the issue of detecting chaos in experimental recordings are discussed. Furthermore,
nonlinear techniques for time series analysis are pointed out, as well as the main quantities that
characterize chaos, namely the maximum Lyapunov exponent and the correlation dimension.
These tools are then employed in Chapter~\ref{chap: chaos analysis} with the objective of analyzing
the chaotic behavior of the Burridge-Knopoff model through the electronic integrated board.














