\chapter*{Conclusions}
\addcontentsline{toc}{chapter}{Conclusions}

%The analysis of nonlinear dynamical systems relies on the possibility of reconstructing proxy state space
%evolutions out of recorded scalar sequences, a possibility granted by Takens' theorem. Time-delay
%embedding is a viable and widespread method to carry out this reconstruction, although its correct
%implementation is not trivial. No definitive answer to the issue of optimal embedding exists.
The analysis of nonlinear dynamical systems depends on the ability to reconstruct proxy state space evolutions from recorded scalar sequences, a capability ensured by Takens' theorem. Time-delay embedding is a common and effective method for achieving this reconstruction, although its proper implementation is not straightforward. Currently, there is no definitive solution to the question of optimal embedding.
However, the method presented in Chapter~\ref{chap: chaos} allows us to find several acceptable
values of the embedding parameters, as opposed to most approaches which provide a single choice for
$m,L$. The criterion that establishes whether a pair is suitable or not is the correlation dimension
$\nu$; more specifically, if a uniformity region in which $\nu$ is practically constant is present,
the embedding pairs that characterize that region are deemed as valid. Consequently, also the maximum
Lyapunov exponent can be estimated in the uniformity region, providing another useful quantity for chaos.

In Chapter~\ref{chap: chaos analysis} this method has been applied to the electronic implementation of
the Burridge-Knopoff model, with encouraging results. Many different configurations have been analyzed;
more specifically, several linear chains of coupled blocks have been implemented, with the number of
blocks ranging from 2 to 25. The procedure has been carried out starting from the signal $W_1$, i.e.
the ``position'' of the first block of the chain. The system resulted to be chaotic in every
configuration $-$ except the single block instance $-$ since a uniformity region has been detected in
each case.

The correlation dimension $\nu$ was then calculated as a function of the number of coupled
blocks. In the 2 blocks case the result was $\nu_2=2.20\pm0.02$, which is partially in accordance with
other estimates found in literature; more specifically, our estimate is slightly larger probably due
to the higher presence of noise. This initial value rapidly increases with the
number of coupled blocks. From 6 to 25 blocks, the correlation dimension fluctuates around a plateau
$\nu_{\text{pl}}=2.56\pm0.01$.
This indicates that at some point the addition of degrees of freedom does not determine an increase
of the dimension of the attractor at which the dynamics converges.

The maximum Lyapunov exponent behaves similarly. For 2 coupled blocks we found $\text{MLE}_2=(54\pm1)$
Hz, in compliance with the numerical results. This value also increases with the number of coupled
blocks, eventually reaching a plateau at about 865 Hz.

Other configurations of the system were then analyzed. More precisely, for the odd values of the
number of coupled blocks (3, 5, 7, \ldots, 25), the embedding procedure was carried out using
$W_k$ as the signal instead of $W_1$, where the $k$-th block is located in the center of the linear
chain $-$ e.g.\ in the 9 blocks case the analyzed signal would be $W_5$. Once again, the system was
found to be chaotic in every configuration, but important discrepancies are present for $\nu$ and MLE\@.

The majority of the estimates for $\nu$ using the center blocks are significantly larger with respect
to the boundary blocks case. This is in apparent contradiction with Takens' theorem, which states that
the dynamics does not depend on the state variable chosen for the analysis. Since this is true for
noiseless, finely sampled and infinitely long sequences, there can be ``experimental'' reasons for
which the theorem does not seem to hold. Indeed, the increase in the correlation dimension can be
explained using the oscilloscope quantization, i.e.\ the number of bits with which the signal is sampled.
We have proven that the correlation dimension increases with the quantization. Since the peaks
reached in the signal $W_k$ are smaller with respect to the signal $W_1$, the quantization is more
relevant in the center block case, resulting in these discrepancies on the estimates of $\nu$.
Considering also the fact that the uncertainties are higher using the center blocks, it is reasonable
to assume that the most suitable estimate for the correlation dimension of the system is the one
obtained using $W_1$ as the time series, i.e.\ $\nu_{\text{pl}}=2.56\pm0.01$.

The situation regarding the maximum Lyapunov exponent is different. Also in this case the estimates using
the center blocks are higher with respect to the boundary block case. However, this cannot be
attributed to the oscilloscope quantization. One hypothesis is the presence of more than
one positive Lyapunov exponent. One of the assumptions of the ``chasing chaos'' algorithm is that
the MLE is the only positive exponent. The method is still able to identify chaos and estimate $\nu$,
but a systematic error on the estimate of MLE is inevitable. Since the MLE reaches two different
plateaux in the two cases, the simplest final estimate of the system's maximum Lyapunov exponent
is given by the average of the two, which yields $\text{MLE}_{\text{pl}}=(1.11\pm0.25)$ kHz. Despite
the very large uncertainty of this value, its order of magnitude complies with the fact that the
characteristic time of the single BK blocks is $\tau=1$ ms, which corresponds to a frequency of 1 kHz.
This means that the circuit keeps the memory of itself for about 1 ms, after which the time evolution
cannot be inferred anymore from the initial condition.

%future developments, tipo fare i cosi bidimensionali e cercare la gutenberg richter

There are several possible future developments which can improve the results found in this work and
better characterize the integrated electronic board. For example, using a method which is able to
distinguish between different positive Lyapunov exponents can give more insights on the chaotic behavior
of the system. Another interesting possibility consists in overcoming the linear block chain, building
different topologies of the systems and analyzing the chaotic behavior of each one; e.g.\ two-dimensional
systems or linear chains with periodic boundary conditions can be implemented and studied. In the end,
it could be fascinating to carry out some ``earthquake simulations'' using the integrated board:
by building a $5\times5$ array of BK blocks, topples and avalanches can potentially be observed.
Furthermore, the compliance with Gutenberg-Richter's law could be verified, e.g.\ by tracking all
25 signals at the same time and recording the number of times that a peak overcomes some threshold.
This would establish conclusively whether the integrated board is a suitable implementation of the
mechanical Burridge-Knopoff model or not.










