\chapter{Chaos in theory and experiments}
\label{chap: chaos}

\section{Dynamical systems}
\label{sec: dynamical systems}

The concept of dynamical systems is quite general, since anything that moves can be considered as a
dynamical system \cite{ref:fractal_dim}. If these changes are driven
by specific rules, we say that the system is deterministic; otherwise, if the rules
are random, it is stochastic. The main feature of chaotic systems is the fact that they are unpredictable
despite being deterministic; in order to better explain the meaning of this statement, it is essential
to involve some mathemathical definitions.

The instantaneous state of a dynamical system is described by a vector $\mathbf{s}$ within a
state space $\mathcal{S}$ (typically, $\mathcal{S}\subseteq \mathbb{R}^M$).
The state vector evolves in time according to an evolution operator $\mathcal{E}_t$ such that:
\begin{equation}
\label{eq: state space evolution}
    \mathbf{s}(t+t_0)=\mathcal{E}_t[\mathbf{s}(t_0)].
\end{equation}

In theoretical systems the state space $\mathcal{S}$ is well-defined and the evolution operator
$\mathcal{E}_t$ is given; in most cases, $\mathcal{E}_t$ is defined by a set of differential equations
which can be solved, analytically or numerically, to find the system's evolution $\mathbf{s}(t)$
out of the initial conditions $\mathbf{s}(0)$.

A dynamical system is defined to be linear if the superposition principle holds, i.e.:
\begin{equation}
\label{eq: linear systems}
    \mathcal{E}_t[c_1\mathbf{s}_1+c_2\mathbf{s_2}]=c_1\mathcal{E}_t[\mathbf{s}_1]+
    c_2\mathcal{E}_t[\mathbf{s}_2].
\end{equation}
If the last equality is not satisfied the system is said to be nonlinear.
Nonlinearity is a necessary (but not sufficient) condition for the system to be chaotic.

In experimentally observed systems the state space is not always fully accessible; moreover,
the evolution operator $\mathcal{E}_t$ is rarely known.
Observing a system consists in recording some signal $y(t)$ out of it through some measurement
function $\mathcal{M}$ acting on the system's state, i.e. $y(t)=\mathcal{M}[\mathbf{s}(t)]$.
This continuous signal is always sampled and digitized, producing a finite time series (or sequence)
$\{y_n\}$ with $n=1,...,\ell$.


\section{Formal definition of chaos}
\label{sec: chaos in theory}

An universally accepted mathematical definition of chaos does not exists but
a commonly used definition is the following, originally formulated by Robert L.
Devaney \cite{ref:chaos_definition}.

Three conditions are necessary and
sufficient to define a system as chaotic: (i) sensitivity to initial conditions;
(ii) topological transitivity;
(iii) density of periodic orbits.

Going into detail:
\begin{itemize}
    \item[(i)] Sensitivity to initial conditions is a property that characterizes chaotic
    systems and makes their evolution hard to predict.

    Given two initial conditions $\mathbf{s}_1(0)$, $\mathbf{s}_2(0)$ that are arbitrarily close within the state
    space ($||\mathbf{s}_1(0)-\mathbf{s}_2(0)||<\varepsilon$),
    the system having sensitive dependence on initial conditions means that
    the two trajectories evolving out of these initial conditions diverge exponentially in time, i.e.
    for large $t$:
    \begin{equation}
    \label{eq: trajectory divergence}
        ||\mathbf{s}_1(t)-\mathbf{s}_2(t)||\propto e^{\lambda t},
    \end{equation}
    where $\lambda$ is called maximum Lyapunov exponent (MLE).
    In order for the system to be chaotic, $\lambda$ has to be positive.

    It is also important that the orbits $\mathbf{s}_1(t)$, $\mathbf{s}_2(t)$ remain bounded at large $t$,
    otherwise, if orbits went to infinity, it would be simple for their distance to diverge exponentially.
    
    The most important consequence of this property is that, as far as we are
    able to precisely measure the initial state of a system, there will always be
    a small error (given for example by measuring instruments) which can grow
    rapidly over time. Therefore, even if we know exactly the deterministic laws
    governing time evolution, our predictions on the behaviour of the system
    after a certain time will no longer be reliable. Furthermore, if the precision
    with which we measure the state of the system in the initial instant is
    improved by a factor of $10$, we only gain a $\log(10)$ factor for the maximum
    time for which the predictions are accurate.

    \item[(ii)] Topological transitivity is the property according to which a chaotic trajectory eventually connects
    any region of the state space with any other. In other words, the state space of a chaotic system
    cannot be decomposed into disjoint subsets.

    \item[(iii)] Density of periodic orbits means that for any given point in the state space
    there is a periodic orbit that passes arbitrarily close to it, i.e. periodic orbits make up a dense set.

\end{itemize}


\section{The issue of detecting chaos}
\label{sec: chaos in experiments}

When dealing with experimentally observed systems, the precise laws that describe the dynamics are unknown.
What is known is the time series, which can be used to assess condition (i) of Devaney's definition
of chaos. Instead, conditions (ii) and (iii) are difficult to identify with the time series only.
However, there are observable consequences. In particular, the time
evolution of a chaotic system in the state space always converges to an object called strange
attractor, characterized by a fractal structure \cite{ref:fractal_dim,ref:fractal_book}. This means that strange
attractors exhibit self-similarity (or self-affinity\footnote{In general, it is more correct to speak
of self-affinity, since in the case of self-similarity the
object is scaled by the same amount in all space directions, but in self-affinity scaling is not
necessary identical in all directions \cite{ref:mandelbrot_fractal}.}) and have a non-integer dimensionality.

To understand what it means to have non-integer dimensionality, suppose
to consider a fractal object with dimension $1 < D < 2$.
We know in general that if we have an object of dimension $D$ (assuming that
it has a mass density), taking an arbitrary point in it and considering an open
ball centered in that point, we can measure the mass contained in the ball as
a function of the radius. For small distances $m(r)\propto r^D$. For the same density
and radius, the fractal ``weighs" more than a line but less then a surface, as if
a dense set (which is the set of the periodic points) had been removed from the
surface.

Self-similarity, instead, is the exhibition of similar patterns at increasingly smaller
scales; in other words, a fractal does not appear simplified when we see it zoomed. Despite strange attractors
also existing in non-chaotic systems \cite{ref:strange_attractors_non_chaotic}, the estimate of the system's non-integer dimension
is often used as a tool to identify chaos.

Another important property allows
us to identify chaos: the trajectory winds around forever never repeating on a
strange attractor and the time series arising by chaotic systems are aperiodic
and characterized by broad, noise-like Fourier spectra \cite{ref:abarbanel_fourier_spectra}.
This also means that linear techniques
such as fast Fourier transform (FFT) applied to sequences cannot distinguish between a chaotic system
and a stochastic one, e.g. a Gaussian white noise source (GWN).

One more necessary requirement to sustain a chaotic flow is that
the number of independent dimensions has to be at least three, assuming no discontinuities \cite{ref:chaos_two_dim_theorem};
this is due to the fact that
with two or less independent variables the trajectory will eventually intersect itself, which
cannot happen due to the aperiodic nature of chaotic dynamics.


\section{The embedding procedure}
\label{sec: embedding}

In experimentally observed systems it is only possible to utilize the finite time series $\{y_n\}$,
with the aim to at least obtain some invariant quantities, e.g. the maximum Lyapunov exponent or
the ``effective" dimension of the system. Under certain conditions, a procedure exists with which
it is possible to reconstruct the entire state space using only one variable, which is a
function of the state space vector $\mathbf{s}(t)$. This procedure is called ``time delay embedding''
\cite{ref:packard1980geometry}.

Embedding consists in building a sequence of $m$-dimensional vectors $\mathbf{Y}_n$ by picking
$m$ time-delayed samples of the sequence $y_n$, i.e.:
\begin{equation}
    \label{eq: embedding}
    \mathbf{Y}_n=\left(y_n,y_{n+L},y_{n+2L},...,y_{n+(m-1)L}\right),
\end{equation}
where the parameter $L$ is an integer number and is called ``lag''. It is equivalent to make use of
a ``causal'' version of Eq. \ref{eq: embedding} in which the chosen samples are $y_{n-kL}$ instead of
$y_{n+kL}$, with $k=0,...,m-1$ \cite{ref:bradley2015nonlinear}. 

It has been proven by Takens \cite{ref:takens2006detecting} and Ma{\~n}{\'e} \cite{ref:mane2006dimension} that
if the parameters $m$ (dimension) and $L$ (lag) are suitably chosen,
the reconstructed state space evolution is topologically identical to the actual state space dynamics.
This means that a ``good” embedding provides a smooth one-to-one map from the
original state space evolution to the reconstructed one, therefore enabling us to estimate the
properties of the original system that are invariant under this mapping.

The two conditions under which Takens' theorem holds regard the embedding parameters $(m,L)$.
The first one requires that $m>2D$, where $D$ is the dimension of the manifold corresponding to the
time evolution of the system, which is not known a priori; this condition can be relaxed to $m>D$ if
estimating correlation dimension \cite{ref:ding1993estimating}.
The second one states that $L$ is not a multiple of the period of any system's orbit \cite{ref:grassberger1991nonlinear}.

Aside from these two minimal conditions, since Takens' theorem is an existence theorem, it does not
give any hint on how to find the best embedding parameters. The reason for this is that as long as the
two conditions are met, every choice for the couple $(m,L)$ is good for reconstruct the dynamics;
however, this is valid only for noiseless, finely sampled and infinitely long sequences. In reality,
the issue of optimal embedding is a very active field in the physics of chaos, and several techniques
have been developed in order to overcome this problem \cite{ref:perinelli2018identification,ref:perinelli2020chasing,ref:casdagli1991state}.

%m PROBLEMS
Considering the dimension $m$, choosing $m\gg1$ ensures, in principle, that $m$ is
greater than $2D$. However, if $m$ is too large the directions orthogonal to the deterministic noiseless
trajectory will be dominated by noise and will not provide extra information; in other words, supposing
that $m_0$ is the minimum dimension for which the system is correctly embedded, the remaining $m-m_0$
dimensions would be populated by noise, being thus redundant and resulting in an increase of the
computational cost. Moreover, the larger is the embedding dimension $m$,
the fewer independent embedding vectors are available, which is an issue considering the finiteness
of the time series.

%L PROBLEMS
The choice on the lag $L$ is also not trivial. If $L$ is too short, the elements of the embedding vectors
$\mathbf{Y}_n$ will be strongly correlated, resulting in all the points being clustered on the
diagonal of the reconstructed space; in presence of noise, therefore, the trajectory would be
indistinguishable from the diagonal itself. On the other hand, a too large $L$ presents the opposite
problem, i.e. the elements of $\mathbf{Y}_n$ to be completely uncorrelated with each other;
this implies that the time evolution is ``blurred'', in the sense that the trajectory is folded over on itself
and the system evolution is lost.


\section{Correlation dimension}
\label{sec: correlation dimension}








\section{Maximum Lyapunov exponent}
\label{sec: mle}







\section{A method for detecting chaos}
\label{sec: method}









\begin{comment}



\end{comment}