\documentclass[a4paper,11pt,aps,secnumarabic,balancelastpage,amsmath,amssymb,nofootinbib,floatfix, table,twoside,openright]{report}
\usepackage{packages}
\usepackage[backend=biber,style=numeric,sorting=none]{biblatex}
\addbibresource{references.bib}


\begin{document}

\pagenumbering{roman}
\subfile{title.tex}

\hypersetup{linkcolor=black}
\tableofcontents
\clearpage

\subfile{0_introduction.tex}
\clearpage

\pagestyle{fancy}
\fancyhead[LE,RO]{\thepage}
\fancyhead[RE,LO]{\textit{\rightmark}}
\cfoot{}
\pagenumbering{arabic}

\subfile{1_electronic_analog.tex}
\clearpage

\subfile{2_chaos.tex}
\clearpage

\subfile{3_chaos_analysis.tex}
\clearpage

\printbibliography
\addcontentsline{toc}{chapter}{Bibliography}


\end{document}




\begin{comment}


CAPITOLO 1: Spiego il Burridge-Knopoff e l'analogo elettronico seguendo il paper (inductorless ecc).
            Faccio vedere l'implementazione sulla breadboard e confronto con la board (single block analysis).

    DA FARE: - Metti foto dell'implementazione breadboard e soprattutto di quella integrata.
             - Risistema il bifurcation.
             - Parla un po' dei terremoti, cioè di perché è importante studiare il BK.


CAPITOLO 2: Parla del caos guardando la tesi di dottorato di Perinelli.
            Non devono esserci cose non spiegate e metti tante ref.

    DA FARE: - Spiega cosa sono nu e mle.


CAPITOLO 3: Spiego l'analisi del caos seguendo il paper (chasing chaos).
            Faccio vedere il caos sulla board accoppiando vari blocchi.
    
    DA FARE: - Parla del metodo di analisi che hai usato.
             - Spiega la nu che aumenta con la quantizzazione dell'oscilloscopio, cioè che aumenta
               sia il valore che l'errore.
             - Di' che l'mle è una limitazione del metodo di calcolo usato, ma che comunque è intorno
               al kHz che ha senso con la costante tempo del circuito; prendi la distanza tra boundary
               e center come indicazione dell'errore sistematico del metodo.
             

\end{comment}